%%%%%%%%%%%%%%%%%%%%%%%%%%%%%%%%%%%%%%%%%%%%%%%%%%%%%%%%%%%%%%%%%%%%%%%%%%%%%%%%
% USENIX-like formatting (approximation) without the usenix .sty file
%%%%%%%%%%%%%%%%%%%%%%%%%%%%%%%%%%%%%%%%%%%%%%%%%%%%%%%%%%%%%%%%%%%%%%%%%%%%%%%%
\documentclass[letterpaper,twocolumn,11pt]{article}

% Page layout similar to USENIX
\usepackage[letterpaper,top=1.5in,bottom=1.7in,left=1.1in,right=1.1in]{geometry}
\setlength{\columnsep}{0.25in}

% Fonts and micro-typography
\usepackage{times}          % USENIX uses a compact, readable serif; Times is close
\usepackage{microtype}
\usepackage{amsmath,amssymb}
\usepackage{url}
\usepackage{tabularx}
\usepackage{graphicx}
\usepackage{booktabs}
\usepackage{array}
\usepackage{filecontents}
\usepackage{enumitem}
\usepackage[numbers,sort&compress]{natbib}

% Tighter spacing for lists and sections
\setlist{nosep, left=0pt}
\setlength{\parskip}{0pt}
\setlength{\parindent}{10pt}
\makeatletter
\renewcommand\section{\@startsection{section}{1}{\z@}%
  {-10pt plus -2pt minus -2pt}{4pt plus 2pt minus 1pt}{\normalfont\large\bfseries}}
\renewcommand\subsection{\@startsection{subsection}{2}{\z@}%
  {-8pt plus -2pt minus -2pt}{3pt plus 1pt minus 1pt}{\normalfont\normalsize\bfseries}}
\makeatother

% Reduce space before/after captions
\setlength{\abovecaptionskip}{6pt}
\setlength{\belowcaptionskip}{6pt}

% Inline bibliography
%-------------------------------------------------------------------------------
% \begin{filecontents}{\jobname.bib}
\begin{filecontents}{refs.bib}
%-------------------------------------------------------------------------------
@inproceedings{Shen2015,
  author = {D. Shen},
  title = {Defeating Samsung KNOX with zero privilege},
  booktitle = {BlackHat USA},
  year = {2015}
}
@inproceedings{Machiry2017,
  author = {M. Machiry and others},
  title = {{BOOMERANG}: Exploiting the semantic gap in {TEE}s},
  booktitle = {NDSS},
  year = {2017}
}
@inproceedings{Cerdeira2020,
  author = {D. Cerdeira and others},
  title = {{SoK}: Understanding the Prevailing Security Vulnerabilities in {TrustZone}-assisted {TEE} Systems},
  booktitle = {IEEE S\&P},
  year = {2020}
}
@inproceedings{Lindenmeier2024,
  author = {T. Lindenmeier and others},
  title = {{EL3XIR}: Systematic fuzzing of Secure Monitors},
  booktitle = {USENIX Security},
  year = {2024}
}
@inproceedings{Busch2024,
  author = {T. Busch and others},
  title = {GlobalConfusion: Type confusion in GlobalPlatform TAs and the ecosystem impact},
  booktitle = {USENIX Security},
  year = {2024}
}
@inproceedings{Chen2017,
  author = {X. Chen and others},
  title = {Downgrade Attack on TrustZone},
  booktitle = {NDSS},
  year = {2017}
}
@inproceedings{Benhani2019,
  author = {E. M. Benhani and others},
  title = {The Security of {ARM TrustZone} in a {FPGA}-Based {SoC}},
  booktitle = {IEEE Trans. Computers},
  year = {2019}
}
@inproceedings{Pinto2019,
  author = {A. Pinto and N. Santos},
  title = {Demystifying {ARM TrustZone}: A comprehensive survey},
  booktitle = {ACM Computing Surveys},
  year = {2019}
}
@inproceedings{Wan2020,
  author = {X. Wan and others},
  title = {{RusTEE}: Developing Memory-Safe {ARM TrustZone} Applications},
  booktitle = {ACSAC},
  year = {2020}
}
@inproceedings{Liljestrand2019,
  author = {H. Liljestrand and others},
  title = {{PAC} it up: Towards pointer integrity using {ARM} pointer authentication},
  booktitle = {USENIX Security},
  year = {2019}
}
@inproceedings{Clements2020,
  author = {A. A. Clements and others},
  title = {Protecting bare-metal embedded systems with privilege overlays},
  booktitle = {IEEE S\&P},
  year = {2017}
}
@inproceedings{Kocher2019,
  author = {P. Kocher and others},
  title = {Spectre attacks: Exploiting speculative execution},
  booktitle = {IEEE S\&P},
  year = {2019}
}
@misc{ProjectZero2017,
  author = {{Google Project Zero}},
  title = {Analysis of {TrustZone} and {QSEE} vulnerabilities},
  howpublished = {Public reports},
  year = {2016--2018}
}

@inproceedings{Ngabonziza2016,
  author = {B. Ngabonziza and others},
  title = {TrustZone Explained: Architectural Features and Use Cases},
  booktitle = {CollaborateCom},
  year = {2016}
}
@inproceedings{Li2021,
  author = {S. Yun and others},
  title = {Automatic Techniques to Systematically Discover New Heap Exploitation Primitives},
  booktitle = {USENIX Security},
  year = {2020}
}
\end{filecontents}

%-------------------------------------------------------------------------------
\begin{document}
%-------------------------------------------------------------------------------

% Title block spanning both columns (similar to USENIX)
\twocolumn[
  \begin{@twocolumnfalse}
    \begin{center}
      {\LARGE \bf Practical Security Analysis of OP-TEE on the Raspberry Pi 4 \par}
      \vspace{10pt}
      {\large
        \begin{tabular}{c}
          {\rm Nathaniel Smith}\\
          smit4351@purdue.edu
        \end{tabular}
        \qquad\qquad
        \begin{tabular}{c}
          {\rm Michael Bialas}\\
          bialas@purdue.edu
        \end{tabular}
      }
      \vspace{6pt}
    \end{center}
    \vspace{6pt}
  \end{@twocolumnfalse}
]

%-------------------------------------------------------------------------------
\section{Introduction}
%-------------------------------------------------------------------------------

ARM TrustZone technology provides hardware-enforced isolation to protect critical
assets like cryptographic keys and biometric data from a compromised Normal World
operating system~\cite{Ngabonziza2016}. This isolation is foundational to modern
device security. Open-source Trusted Execution Environments (TEEs), such as
OP-TEE (Open Portable TEE), are widely used to implement the Secure World
software stack. However, the security of any TEE relies critically on the
correct hardware integration by the System-on-Chip (SoC) vendor~\cite{Benhani2019}.

The Raspberry Pi 4 (RPi4), our target platform, presents an important case
study. While its BCM2711 SoC features TrustZone-capable processors, its
peripheral isolation and DMA protections are known to be less comprehensive than
those in high-end consumer devices~\cite{Benhani2019, Cerdeira2020}. This
potentially incomplete hardware protection creates significant attack vectors.
An adversary who has compromised the Normal World kernel (e.g., Linux) could
leverage misconfigured DMA-capable peripherals to read or write Secure World
memory, completely bypassing TrustZone's software-level security guarantees.

This project is investigating the practical security of OP-TEE running on the
Raspberry Pi 4. We are focusing on two primary attack vectors identified in our
proposal:
\begin{enumerate}
    \item \textbf{DMA Attack:} A hands-on attack leveraging a Normal World kernel driver to program an RPi4 peripheral to perform Direct Memory Access (DMA) into Secure World memory.
    \item \textbf{SMC Fuzzing:} Systematically fuzzing the Secure Monitor Call (SMC) interface of the OP-TEE monitor, which is the boundary between the Normal and Secure Worlds~\cite{Lindenmeier2024}.
\end{enumerate}

A compromise of the TEE, designed to be the root of trust, would invalidate the
device's core security guarantees. This could enable the theft of cryptographic
keys, bypassing of digital rights management (DRM), or the insertion of
persistent, undetectable malware. Therefore, analyzing the practical security of
real-world TEE implementations on popular hardware is a critical research area.

Despite the RPi4's popularity, its TrustZone security and peripheral
configuration are not as well-documented or analyzed as mainstream mobile SoCs.
Publicly disclosed vulnerabilities show that flaws at both the hardware
integration and software monitor levels can compromise TrustZone's security
guarantees~\cite{Shen2015, ProjectZero2017}. This project aims to provide a
concrete, hands-on analysis of these risks on a widely available development
platform.

\subsection*{Contributions}
This project update presents our current progress toward the following
contributions:
\begin{itemize}
  \item Ongoing work to establish a \textbf{Raspberry Pi 4 testbed} for OP-TEE, including working through the complex build and boot process.
  \item Preliminary research into the \textbf{RPi4's DMA-capable peripherals} by analyzing SoC documentation and existing kernel sources.
  \item A defined methodology for \textbf{SMC interface fuzzing} that we plan to implement once the testbed is stable.
\end{itemize}

\subsection*{Threat model and scope}
Our threat model assumes an adversary has achieved kernel-level access in the
Normal World on the RPi4 (e.g., by exploiting a Linux kernel vulnerability).
From this position, the adversary can invoke SMC instructions to probe the
OP-TEE monitor and, crucially, attempt to program DMA-capable peripherals on the
BCM2711 SoC~\cite{Benhani2019}. We focus on attacks that do not require
physical access or hardware modification~\cite{Benhani2019, Clements2020}. Our
investigation is focused entirely on the RPi4 platform running OP-TEE.

\subsection*{Organization}
Section~\ref{sec:background} provides background on TrustZone architecture and
OP-TEE. Section~\ref{sec:methodology} describes our hands-on analysis
methodology, including our current work on the testbed setup, the plan for the
DMA attack, and our fuzzing strategy. Section~\ref{sec:results} presents our
preliminary results. Section~\ref{sec:related} discusses related work.

%-------------------------------------------------------------------------------
\section{Background}
\label{sec:background}
%-------------------------------------------------------------------------------

\subsection{ARM TrustZone Architecture}

ARM TrustZone technology partitions processor execution into two domains: the
Normal World for general-purpose operating systems (like Linux) and the Secure
World for trusted code~\cite{Ngabonziza2016}. The architecture defines four
Exception Levels (ELs), from EL0 (user-space) to EL3 (the most privileged
level). The Secure Monitor, running at EL3, acts as the gatekeeper between the
Normal World (which typically uses EL0, EL1, and EL2 for hypervisors) and the
Secure World (which uses S-EL0 and S-EL1)~\cite{Pinto2019}. This separation
extends to memory and peripherals through a special security bit (NS-bit). The
Normal World requests services from the Secure World via the Secure Monitor
Call (SMC) instruction, which traps to the EL3 monitor~\cite{Pinto2019}.

Memory isolation is enforced by hardware like the TrustZone Address Space
Controller (TZASC)~\cite{Pinto2019}. However, this protection can be bypassed by
DMA-capable peripherals if they are not ``TrustZone-aware'' or are misconfigured.
A non-secure peripheral programmed by the Normal World kernel could potentially
initiate a DMA transaction targeting a physical address in Secure World memory,
violating the isolation boundary~\cite{Benhani2019}.

\subsection{OP-TEE and the Raspberry Pi 4}
Our work targets OP-TEE, an open-source TEE implementation that serves as the
Secure World OS (running at S-EL1) and Secure Monitor (at EL3). It is designed
to be portable across various ARM platforms.

The Raspberry Pi 4 uses the Broadcom BCM2711 SoC. While this chip supports
TrustZone, its documentation regarding the configuration of peripheral and
memory protections is limited. This lack of clear documentation, combined with
the RPi4's design as a development board, makes it an ideal platform for
investigating potential hardware integration weaknesses. We are working on
building and running OP-TEE from source, which will allow us to analyze and
modify its behavior.

\subsection{Vulnerability Landscape}

Publicly disclosed TrustZone vulnerabilities show recurring patterns~\cite{Cerdeira2020}.
On the software side, SMC handlers are vulnerable to memory safety issues
(buffer overflows, integer overflows) and logical flaws (TOCTOU, authentication
bypasses)~\cite{Machiry2017, Lindenmeier2024}. On the hardware side, DMA
attacks from misconfigured peripherals are a well-known threat~\cite{Benhani2019}.
Our project is investigating both of these angles on our specific target platform.

%-------------------------------------------------------------------------------
\section{Methodology}
\label{sec:methodology}
%-------------------------------------------------------------------------------

Our investigation employs a hands-on, experimental approach on physical RPi4 hardware.
This is critical for probing hardware-level vulnerabilities that would not be present
in an emulated environment.

\subsection{RPi4 Testbed Setup}

Our first and most critical step is creating a stable and debuggable testbed.
This is our primary focus at present, as we do not yet have OP-TEE fully running.
This is a complex process that involves several stages we are currently working through:
\begin{enumerate}
    \item \textbf{Hardware Acquisition:} We have secured a Raspberry Pi 4 Model B and a USB-to-UART serial adapter for console access.
    \item \textbf{OP-TEE Compilation:} We are working to set up the cross-compilation toolchain and resolve build dependencies for the OP-TEE source code. This involves building the monitor (EL3), the TEE OS (S-EL1), and trusted applications.
    \item \textbf{Bootloader and Kernel Configuration:} We are investigating the RPi4 bootloader (which involves U-Boot) and the Linux kernel modifications needed to boot in a TrustZone-enabled state. This is a non-trivial task involving configuring the boot chain to load OP-TEE correctly, and modifying the Linux kernel's device tree to reserve secure memory and inform the kernel of the TEE's presence. We are actively debugging this hand-off process.
    \item \textbf{Debug Interface:} We have established a serial console connection to view boot-time logs, which is essential for debugging the boot-up process.
\end{enumerate}
This setup, once complete, will provide a platform where we control the entire software stack.

\subsection{DMA Attack Development Plan}

Once the testbed is operational, our plan for the DMA attack is as follows:
\begin{enumerate}
    \item \textbf{Hardware Fingerprinting:} Once the testbed is stable, we will begin by analyzing the BCM2711's memory-mapped I/O (MMIO) space and Linux drivers to confirm the register interfaces for known DMA-capable peripherals.
    \item \textbf{Peripheral Identification:} We will select the most promising peripheral for an attack. The XHCI (USB) and network controllers are prime candidates, as they are complex and designed for high-performance memory access.
    \item \textbf{Memory Mapping:} We will precisely identify the physical memory regions allocated to OP-TEE's Secure RAM by analyzing the boot logs and device tree configuration.
    \item \textbf{PoC Kernel Module:} We plan to develop a custom Linux kernel module. This module will run in the Normal World and use our findings from step 1 to program the chosen peripheral's DMA engine to attempt a write to a known physical address within Secure RAM. The initial goal is to establish a basic ``read/write primitive.''
    \item \textbf{Targeted Attack:} If a primitive is established, we will then attempt to refine the attack to target a specific, high-value function, such as the OP-TEE logic responsible for verifying Trusted Application signatures.
\end{enumerate}

\subsection{SMC Interface Fuzzing Plan}
Our second planned attack vector targets the EL3 monitor's software interface. We plan to build an on-device fuzzer from the Normal World.
\begin{itemize}
    \item The fuzzer will generate mutated inputs (function IDs, parameter values) for the SMC instruction.
    \item We will monitor the Secure World's UART console output for any crashes, panics, or anomalous behavior, which would indicate a potential vulnerability~\cite{Lindenmeier2024}.
\end{itemize}
This phase will begin after we make significant progress on the testbed setup and DMA attack.

%-------------------------------------------------------------------------------
\section{Preliminary Results}
\label{sec:results}
%-------------------------------------------------------------------------------

\subsection{Testbed Setup Progress}

Our work is currently focused on getting the OP-TEE testbed running. We do not
have a fully operational boot of OP-TEE yet, but we have made progress on the components:
\begin{itemize}
    \item We have the Raspberry Pi 4 hardware and have successfully established a serial console for debugging, which allows us to see bootloader and kernel logs.
    \item We have set up a cross-compilation toolchain and are working on compiling the various parts of OP-TEE and a compatible Linux kernel.
    \item We are actively working to resolve issues with the bootloader configuration and the kernel hand-off to the OP-TEE monitor. This is proving to be the most challenging part of the setup, as documentation for this specific combination is sparse.
\end{itemize}

\subsection{DMA Controller Investigation}

While the testbed is not yet running OP-TEE, we have begun the preliminary research for the DMA attack.
\begin{itemize}
    \item We have been analyzing the Linux kernel device tree and source code for the RPi4 to identify the physical addresses and driver interfaces for high-speed, DMA-capable peripherals.
    \item The XHCI (USB 3.0) controller appears to be a promising target. We are studying its driver to understand how it configures and triggers DMA operations.
\end{itemize}

\subsection{SMC Fuzzing Status}
The SMC fuzzing harness is not yet implemented. Our full attention is currently on establishing the testbed. This phase of the project is planned for later.

\begin{table}[h]
\centering
\begin{tabularx}{\columnwidth}{@{}l c X@{}}
\toprule
\textbf{Component} & \textbf{Status} & \textbf{Notes} \\
\midrule
RPi4 Hardware & Acquired & RPi4 Model B, UART adapter \\
OP-TEE Build & In Progress & Resolving compile/boot issues \\
Debug Setup & Operational & Serial console is working \\
DMA Attack PoC & Researching & Identifying peripheral targets \\
SMC Fuzzer & Not Started & Planned for next phase \\
\bottomrule
\end{tabularx}
\caption{Summary of project status update.}
\end{table}

%-------------------------------------------------------------------------------
\section{Related Work}
\label{sec:related}
%-------------------------------------------------------------------------------

Research on TrustZone security has advanced significantly. Early work demonstrated
the feasibility of cross-boundary attacks~\cite{Shen2015}, proving that TrustZone's
security depends on correct implementation.

Our work on SMC fuzzing builds on recent advances in automated analysis of TEEs.
Researchers have used both static analysis~\cite{Machiry2017} and dynamic
analysis/fuzzing~\cite{Lindenmeier2024} to find vulnerabilities in TEEs and Secure
Monitors. Our plan is to apply these dynamic techniques to the specific OP-TEE
implementation on the RPi4.

Our DMA attack methodology is informed by a body of work on hardware-level security.
Multiple studies have examined how incomplete hardware protections, misconfigured
bus fabrics, or vulnerable peripherals create attack surfaces that bypass software
safeguards~\cite{Benhani2019}. These studies often reveal wide variations in
protection quality across devices~\cite{Cerdeira2020}. Our work applies these
concepts directly to the RPi4, which has not been as extensively studied in this
context as other consumer devices.

Additional research has examined trusted application vulnerabilities~\cite{Busch2024},
downgrade attacks~\cite{Chen2017}, and memory safety issues~\cite{Li2021}. This
project complements this existing work by focusing on the practical, hands-on
exploitation of a specific, popular, and accessible hardware platform.

%-------------------------------------------------------------------------------
\bibliographystyle{plainnat}
%\bibliography{\jobname}
\bibliography{refs}

%%%%%%%%%%%%%%%%%%%%%%%%%%%%%%%%%%%%%%%%%%%%%%%%%%%%%%%%%%%%%%%%%%%%%%%%%%%%%%%%
\end{document}
